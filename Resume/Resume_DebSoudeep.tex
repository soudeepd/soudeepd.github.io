%%%%%%%%%%%%%%%%%%%%%%%%%%%%%%%%%%%%%%%%%
% Medium Length Professional CV
% LaTeX Template
% Version 2.0 (8/5/13)
%
% This template has been downloaded from:
% http://www.LaTeXTemplates.com
%
% Original author:
% Trey Hunner (http://www.treyhunner.com/)
%
% Important note:
% This template requires the resume.cls file to be in the same directory as the
% .tex file. The resume.cls file provides the resume style used for structuring the
% document.
%
%%%%%%%%%%%%%%%%%%%%%%%%%%%%%%%%%%%%%%%%%

%----------------------------------------------------------------------------------------
%	PACKAGES AND OTHER DOCUMENT CONFIGURATIONS
%----------------------------------------------------------------------------------------

\documentclass{resume} % Use the custom resume.cls style

\usepackage[left=0.7in,top=0.8in,right=0.7in,bottom=0.8in]{geometry} % Document margins
\usepackage{titlesec}
\usepackage{color,hyperref,xcolor}
\hypersetup{colorlinks=true,urlcolor=blue}
\newcommand{\tab}[1]{\hspace{.2\textwidth}\rlap{#1}}

\name{Soudeep Deb} % Your name at the top

% If you don't want one of the addresses, simply remove all the text in the first or second \address{} bracket

\address{{\bf} \\  1221 Avenue of the Americas, New York, NY 10020 \\ Phone: +1(212)664-4519 \\ Email: soudeep.deb@nbcuni.com}  % Your address 1\\

%\address{{\bf E-mail} \\ swati11289@uchicago.edu} % Your address 2
%\titlespacing\section{0pt}{9pt plus 2pt minus 1pt}{0pt plus 1pt minus 1pt}
%\titlespacing\subsection{0pt}{9pt plus 2pt minus 1pt}{0pt plus 1pt minus 1pt}
%\titlespacing\subsubsection{0pt}{9pt plus 2pt minus 1pt}{0pt plus 1pt minus 1pt}

\begin{document}



%----------------------------------------------------------------------------------------
%	SUMMARY SECTION
%----------------------------------------------------------------------------------------

\begin{rSection}{Summary}

\begin{rSubsection}{}{}{}{}
\item {\bf Senior Lead Data Scientist} at NBC Universal Media, LLC.
\item 10+ years of training and experience in Statistics and Data Science.
\item Expertise in statistical inference, data analysis, and developing statistical methodologies for real life problems.
\item Primary areas of research are time series data, forecasting, spatial modeling, and sports analytics.
\item Proficient in learning new procedures and taking lead in diverse projects.
\end{rSubsection}
\end{rSection} 



%----------------------------------------------------------------------------------------
%	EXPERIENCE SECTION
%----------------------------------------------------------------------------------------


\begin{rSection}{Professional Experience}
%------------------------------------------------

\begin{rSubsection}{NBC Universal Media, LLC, New York, NY, USA.}{{Sep 2018 - present}}{}{}
\item {Position}: Senior Lead Data Scientist. 
\item {Responsibility}: Modeling time series data for TV shows, Provide forecast for programs from different channels, and to gather insights about viewership and ratings.
\end{rSubsection}

\end{rSection}


%----------------------------------------------------------------------------------------
%	EDUCATION SECTION
%----------------------------------------------------------------------------------------

\begin{rSection}{Education}
\begin{rSubsection}{University of Chicago}{{Sep 2013 - Aug 2018}}{}{}
\item {PhD in Statistics} 
\end{rSubsection}
\begin{rSubsection}{Indian Statistical Institute}{{Jul 2008 - Jun 2013}}{}{}
\item  {Master of Statistics (Hons.) with distinction} \hfill Total percentage score: 81.0\% 
\item  {Bachelor of Statistics (Hons.) with distinction} \hfill Total percentage score: 83.4\%
\end{rSubsection}

\end{rSection}



\begin{rSection}{Other Experience}
%------------------------------------------------

\begin{rSubsection}{The Alan Turing Institute, London, United Kingdom.}{Dec 2017}{}{}
\item {Position}: Delegate for the Data Study Group. 
\item {Project}: Geospatial time-series analyses to predict demand for a global satellite communications network.
\end{rSubsection}

\begin{rSubsection}{Instituto de Pesquisa Ambiental do Amaz\^ onia, (IPAM), Bras\'{i}lia, Brazil.}{{Jun - Aug 2016}}{}{}
\item {Position}: Summer intern. 
\item {Project}: Hydropower Construction and Deforestation in the Tapaj\'{o}s River Basin: Linking Forest Cover to Changes in Water Balance.
%\item {Details}: 
\end{rSubsection}

\begin{rSubsection}{Eidgenossische Technische Hochschule (ETH), Zurich, Switzerland.}{{May - Jul 2013}}{}{}
\item {Position}: Summer research intern. 
\item {Project}: Moment-Closure Approximations for Mass-action Models in Chemical Kinetics.
%\item {Details}: In chemical kinetics, using the chemical master equation, a system of differential equations can be obtained but solving it is not possible in general since the system is not closed. In this study, we improve existing methods to use moment closure approximations under normality assumption to solve the system for any order moment of the state at any time-point. Then, we devised new methods to do the same under Poisson assumption.
\end{rSubsection}

\begin{rSubsection}{Ministry of Statistics and Programme Implementation, Govt. of India.}{{May 2012}}{}{}
\item {Position}: Team member. 
\item {Project}: Forecasting of Foreign-tourist Arrivals in India.
%\item {Details}: 
\end{rSubsection}

\begin{rSubsection}{Johns Hopkins University, Baltimore, United States of America.}{{May - Jul 2011}}{}{}  
\item {Position}: Summer research intern. 
\item {Project}: Estimating the genetic relationship between two random individuals from genome sequence data.
%\item {Presented in}: Young Statisticians Conference 2013, Melbourne, Australia; Conference on Contemporary Issues and Applications of Statistics (CIAS 2012), Kolkata, India.
%\item {Details}:  In this study, we describe two methods to estimate IBD probabilities from the genome sequence data of two random individuals depending on the variability of heterozygosity and thereby predict the relationship between them. We applied all our theoretical methods to a simulated dataset to show that the methods are appropriate. Later, the methods were applied to a real-life dataset to find the relationship between individuals.
\end{rSubsection}

\end{rSection}




%----------------------------------------------------------------------------------------
%	AWARDS & ACHIEVEMENTS SECTION
%----------------------------------------------------------------------------------------
\begin{rSection}{Awards \& Achievements}
%------------------------------------------------
\begin{rSubsection}{}{}{}{}
\item Recipient of International House Ralph W. Nicholas Fellowship Award, University of Chicago. \hfill {2017-18}
\item Recipient of Graduate Council Travel Grant, University of Chicago. \hfill {2017}
\item Senior Consultant Award at the Department of Statistics, University of Chicago. \hfill {2016-17}
\item Selected among 30 students from India for International Mathematical Olympiad Training Camp. \hfill {2007, 2008}
\item Recipient of Kishore Vaigyanik Protsahan Yojana scholarship, Indian Institute of Science. \hfill {2007 - 2013}
\end{rSubsection}
\end{rSection}





\begin{rSection}{Technical Strengths}
\begin{rSubsection}{}{}{}{}
\item {\bf Proficient:} R, MATLAB, \LaTeX, Microsoft Office.
\item {\bf Learning:} Python, SQL.
\end{rSubsection}
\end{rSection}



%----------------------------------------------------------------------------------------
%      PUBLICATIONS SECTION
%----------------------------------------------------------------------------------------
\begin{rSection}{Publications and ongoing research}
\begin{rSubsection}{}{}{}{}
    %\item {\bf Deb, S.}, Wu, W. B.; Clustering of Time Series Data using Spectral Density Estimates; In preparation.
    \item {\bf Deb, S.}, Tsay, R. S.; Spatio-temporal Models with Space-time Interaction and Their Applications to Air Pollution Data; Under revision, Statistica Sinica, Preprint: https://arxiv.org/abs/1801.00211.
    \item {\bf Deb, S.}, Dey, D. (2017); Spatial Modeling of Shot Conversion in Soccer to Single out Goalscoring Ability; Under revision, Journal of Sports Analytics, Preprint: https://arxiv.org/abs/1702.05662.
    \item {\bf Deb, S.}, Pourahmadi, M., Wu, W. B. (2017); An Asymptotic Theory for Spectral Analysis of Random Fields;  Electronic Journal of Statistics, Vol. 11, No. 2, p. 4297-4322.
    \item {\bf Deb, S.} (2017); VAR model based clustering method for multivariate time series data; In XXXIV. International Seminar on Stability Problems for Stochastic Models, p. 28.
    \item Chazin, H., {\bf Deb, S.}, Falk, J., Srinivasan, A. (2017); New Statistical Approaches to Intra-individual Isotopic Analysis and Modeling Birth Seasonality in Studies of Herd Animals; {To appear}, Archaeometry.
    %\item Badrinathan, S., {\bf Deb, S.}; Representation in Indian Politics : People's Priorities and Their Effect on Legislative Activity; In preparation.
     \item Prickett, K.C., Guiterrez, C., {\bf Deb, S.}; U.S. Family Firearm Ownership and Firearm-Related Child Mortality from 1976 to 2014, To appear, Pediatrics.
    \item Zechner, C., {\bf Deb, S}., Koeppl, H. (2013); Marginal Dynamics of Stochastic Biochemical Networks in Random Environments; In Control Conference (ECC), 2013 European, p. 4269-4274, IEEE.
    \item Ghosh, S., {\bf Deb, S.} (2013), A Clustering Approach for Mapping Rare Variants Based in Mutual Association. Human Heredity, Vol. 76, No. 2, pp. 98-98. 
\end{rSubsection}
\end{rSection}

%
%%----------------------------------------------------------------------------------------
%%      PROJECTS SECTION
%%----------------------------------------------------------------------------------------
%\begin{rSection}{Other Projects}
%
%\begin{rSubsection}{}{}{}{}
%\item Correlation of Histopathologic Staging in Pancreatitis using Islet Isolation and Transplantation. \hfill {Mar - May 2015}
%\item Association Analyses for Identifying Rare Genetic Variants. \hfill {Jul 2012 - Jun 2013}
%\item Study of Effects of Nano-particles on Drossophila Genes from DNA Microarray Analysis. \hfill {May - Aug 2010}
%\item Analysis of Real-life Data to Determine Groundwater-dependence of People of West Bengal. \hfill {Jan - May 2010}
%\end{rSubsection}
%
%\end{rSection}


%----------------------------------------------------------------------------------------
%	WORK EXPERIENCE SECTION
%----------------------------------------------------------------------------------------


%\begin{rSection}{Teaching Experience}
%
%\begin{rSubsection}{}{}{}{}
%\item Statistics instructor for the summer course in \href{https://caap.uchicago.edu/}{Chicago Academic Achievement Program}.\hfill {\it 2015, 2017}
%\item \href{http://galton.uchicago.edu/courseinfo/courses/2015/win/ann/w23400-3.shtml}{Lecturer} for Stat 234 : Undergraduate Introductory Statistics at University of Chicago. \hfill {\it Winter 2015}
%\item Teaching assistant for several Statistics courses, at University of Chicago. \hfill {\it 2014 - present}
%\item Consulting service to academicians from Biostatistics, Ecology, Anthropology, Medicine.   \hfill {2013 - present}
%\item Team member on a project titled {\it Forecasting of Foreign-tourist Arrivals in India} for {\bf Ministry of Statistics and Programme Implementation, Govt. of India}. \hfill {\it Summer 2012}
%\end{rSubsection}
%
%\end{rSection}

%\begin{rSection}{M.Stat Dissertation}

%\begin{rSubsection}{}{}{}{}
%\item {Project title}: Association Analyses for Identifying Rare Genetic Variants \hfill {Jul 2012 - Jun 2013}
%\item {Details}: In genetics, an alternative paradigm that is becoming increasingly popular is that {\it missing heritability} can be explained by rare variants. In this study, we develop a new method to discover the rare genetic variants responsible for a disease. Then, we compared it with the existing methods to show that our method has greater power.
%\end{rSubsection}

%\end{rSection}





%----------------------------------------------------------------------------------------
%	TECHNICAL STRENGTHS SECTION
%----------------------------------------------------------------------------------------

%----------------------------------------------------------------------------------------
%	LEADERSHIP & COMMUNITY INVOLVEMENT SECTION
%----------------------------------------------------------------------------------------
\begin{rSection}{Other Information}
%\begin{rSubsection}{University of Chicago}{Chicago, Illinois}{}{}
\begin{rSubsection}{Languages}{}{}{} 
\item Fluent: English, Bengali, Hindi.
\item Working knowledge: Urdu, Portuguese.
\end{rSubsection}


\begin{rSubsection}{Positions of Responsibility}{}{}{} 
\item Student Photographer, Department of Statistics, University of Chicago.
\item Event Chief, Annual techno-cultural fest {\it Integration 2013}, Indian Statistical Institute.
\item Student Representative, Cultural Committee, Indian Statistical Institute.
\item Literary Affairs Committee Convener, Boys' Hostel, Indian Statistical Institute.
%\item Joint co-editor, School Magazine {\it Canvas}, Ramakrishna Mission Vidyalaya, Narendrapur.
\end{rSubsection}


%\begin{rSubsection}{Extra-curricular Activities}{}{}{}
%\item Chief Editor for \href{www.sports-nova.com}{Sports-nova}, a sports website based in India.
%\item Previously a regular columnist for \href{http://mufclatest.com/author/soudeepd/}{MUFClatest}, a UK based soccer website.
%\item Intramural Chess Champion in 2016/17 season, at University of Chicago.
%\item Intramural Table Tennis Champion in 2016/17 season, at University of Chicago.
%\item Member in both soccer and cricket team in school and college.
%\item Silver medal in All India Painting Competition (2005), by {\it Young Indians, Mumbai}.
%\item {\it Other activities}: Photography, Blogging, Solving and creating puzzles, Playing bridge, Quizzing.
%\end{rSubsection}

\end{rSection}

\rule{\textwidth}{0.4pt}


%----------------------------------------------------------------------------------------





%----------------------------------------------------------------------------------------
%	EXAMPLE SECTION
%----------------------------------------------------------------------------------------

%\begin{rSection}{Section Name}

%Section content\ldots

%\end{rSection}

%----------------------------------------------------------------------------------------

\end{document}
